%\usepackage{hyperref} %do bibliografii
\raggedbottom %do usuwania odstepow
\usepackage{xargs}                      % Use more than one optional parameter in a new commands

\usepackage[pdftex,dvipsnames]{xcolor}  % Coloured text etc.
\usepackage{xcolor}

\usepackage{polski}
\usepackage{amsfonts}
\usepackage{amsthm}
\usepackage{amsmath}
\usepackage{amssymb} % To pakiet z dodatkowymi symbolami matematycznymi
\usepackage[utf8]{inputenc}
\usepackage{color}
\usepackage{graphicx}
\usepackage{caption}
\usepackage{subcaption} %do podpisów obrazków
\usepackage{bbm}
\usepackage{enumerate}
\usepackage{enumitem}  % zmiana stylu wypunktowania
\usepackage{setspace}
\usepackage{ulem}  % podkreślenia
\onehalfspacing
% \newcommand{\qedwhite}{\hfill \ensuremath{\Box}}  komenda na bialy kwadracik

\allowdisplaybreaks

\newcommand{\R}{\mathbb{R}}

\theoremstyle{plain}
\newtheorem{tw}{\bfseries Twierdzenie}[chapter]
\newtheorem{lem}[tw]{\bfseries Lemat}
\theoremstyle{definition}
\newtheorem{defi}[tw]{\bfseries Definicja}
\newtheorem{wn}[tw]{\bfseries Wniosek}
\newtheorem{uw}[tw]{\bfseries Uwaga}
\newtheorem{ob}[tw]{\bfseries Obserwacja}
\newtheorem{za}[tw]{\bfseries Zadanie}
\newtheorem{pr}[tw]{\bfseries Przykład}


\usepackage{geometry}
\newgeometry{lmargin=3.3 cm, rmargin=3.3 cm}

\numberwithin{section}{chapter}

\renewcommand{\figurename}{\bfseries{Rys.}}


\let\StandardTheFigure\thefigure
\renewcommand{\thefigure}{\thetw}


%robi zeby nie bylo akapitow przed twierdzeniami itp
\makeatletter
\let\sv@thm\@thm
\def\@thm{\let\indent\relax\sv@thm}
\makeatother

\makeatletter

  
%usuwa nazwa rozdzial w spisie tresci

\renewcommand\tocchapter[3]{%
  \indentlabel{\@ifnotempty{#2}{\ignorespaces#2.\quad}}#3%
}



% Make chapter lines of TOC be BOLD
\def\l@chapter{\@tocline{0}{8pt plus1pt}{0pt}{}{\bfseries}}


\makeatother

%Dodaje kropki w spisie tresci

\makeatletter
\def\@tocline#1#2#3#4#5#6#7{\relax
  \ifnum #1>\c@tocdepth % then omit
  \else
    \par \addpenalty\@secpenalty\addvspace{#2}%
    \begingroup \hyphenpenalty\@M
    \@ifempty{#4}{%
      \@tempdima\csname r@tocindent\number#1\endcsname\relax
    }{%
      \@tempdima#4\relax
    }%
    \parindent\z@ \leftskip#3\relax \advance\leftskip\@tempdima\relax
    \rightskip\@pnumwidth plus4em \parfillskip-\@pnumwidth
    #5\leavevmode\hskip-\@tempdima #6\nobreak\relax
    \ifnum#1<0\hfill\else\dotfill\fi\hbox to\@pnumwidth{\@tocpagenum{#7}}\par
    \nobreak
    \endgroup
  \fi}
\makeatother

%ta formula edytuje styl twierdzenia tak ze pogrubia nazwy i sa spoko wyswietlane

\makeatletter
\def\th@plain{%
  \thm@notefont{}% same as heading font
  \itshape % body font
}
\def\th@definition{%
  \thm@notefont{}% same as heading font
  \normalfont % body font
}
\makeatother

\usepackage{todonotes}
\usepackage{xargs}                      % Use more than one optional parameter in a new commands

% \newcommand{\tip}

% \newcommand{\R}{\mathbb{R}}
% \newcommand{\Z}{\mathbb{Z}}
% \newcommand{\N}{\mathbb{N}}
% \newcommand{\C}{\mathbb{C}}
% \newcommand{\protip}

\newcommandx{\tip}[2][1=]{\todo[linecolor=orange,backgroundcolor=SpringGreen,bordercolor=PineGreen,inline,#1]{#2}}
\newcommandx{\protip}[2][1=]{\todo[linecolor=orange,backgroundcolor=Goldenrod,bordercolor=Dandelion,inline,#1]{#2}}


\setlist[enumerate,1]{}%  if you just want to change some aspect of the default enumerate environment
% https://tex.stackexchange.com/questions/94755/a-new-environment-based-on-enumerate
\newlist{conds}{enumerate}{1}
\setlist[conds,1]{label=(\arabic*)}

\newlist{ads}{enumerate}{1}
\setlist[ads,1]{label=Ad. (\arabic*)}




\renewcommand\thetable{\thechapter\arabic{table}}
\renewcommand{\thetw}{\arabic{chapter}.{\arabic{tw}}}
\setlength{\textheight}{21cm}
\setlength{\topmargin}{-1cm}

\bibliographystyle{plain}

\DeclareMathOperator{\interior}{Int}

% new



\newtheoremstyle{zadanie}% name
  {15pt}% space above
  {\topsep}% space below
  {}% body font
  {}% indent amount
  {\bfseries}% theorem head font
  {}% punctuation after theorem head
  {.5em}% space after theorem head
  {\thmname{#1}\thmnumber{ #2}\thmnote{ (#3)}}% theorem head spec
\theoremstyle{zadanie}
\newtheorem{zad}{Zad.}[chapter]

\newtheoremstyle{rozwiazanie}% name
  {\topsep}% space above
  {\topsep}% space below
  {}% body font
  {}% indent amount
  {\bfseries}% theorem head font
  {:}% punctuation after theorem head
  {.5em}% space after theorem head
  {\thmname{#1}\thmnumber{ #2}\thmnote{ (#3)}}% theorem head spec
\theoremstyle{rozwiazanie}
\newtheorem*{rozw}{Rozwiązanie}

\newtheorem*{odp}{Odpowiedź}
% end new